\documentclass{article}
\usepackage{graphicx}

\usepackage[inline]{enumitem}
\usepackage{amsmath}
\usepackage{hyperref}

\newcommand{\e}[1]{{\mathbb E}\left[ #1 \right]}
\newcommand{\enn}[1]{{\mathbb E}_n\left[ #1 \right]}
\newcommand{\Var}[1]{\mathrm{Var}\left[ #1 \right]}
\usepackage{natbib}


\title{Technical details for the SEIR graphical model of the Open SEIR-Graph project}
\author{Oliver Hines}
\date{\today}

\begin{document}
    \maketitle
\section{SEIR graphical model}
    We consider $N$ nodes on the graph, $\mathcal{G}$ indexed by an integer $i \in \{1,...,N\}$. Each node may be in one of four states,
    Susceptible (0), Exposed (1), Infectious(2) or Removed(3). Denoting the state of any particular node by $X_{i,t}$
    for discrete time index, $t\geq 0$, then we define the four time series
    \begin{align*}
        \{S_t\}_{t\geq 0} &= \sum_{i=1}^N \mathcal{I}\{X_{i,t} = 0\} \\
        \{E_t\}_{t\geq 0} &= \sum_{i=1}^N \mathcal{I}\{X_{i,t} = 1\} \\
        \{I_t\}_{t\geq 0} &= \sum_{i=1}^N \mathcal{I}\{X_{i,t} = 2\} \\
        \{R_t\}_{t\geq 0} &= \sum_{i=1}^N \mathcal{I}\{X_{i,t} = 3\}
    \end{align*}
    Where \mathcal{I} is the indicator function that the argument is true. Since the state is categorical, we have that
    at every time point $S_t + E_t + I_t + R_t = N$.

    Starting in the intial configuration $X_{0,0} = 2$ and $X_{i,0}=0$ for $i\neq 0$, the dynamics proceeds in the
    manner below, where we use $n_{i,t}(x,\mathcal{G})$ to denote the number of neighbors of the node $i$, which are in a state $x$ at time $t$.

    At each time point $T$, iterate through each node $i$:
    \begin{itemize}
        \item If $X_{i,T} = 0$, simulate $A \sim \text{Binomial}(n_{i,T}(1,\mathcal{G}),p_E)$ and
              $B \sim \text{Binomial}(n_{i,T}(2,\mathcal{G}),p_I)$, then if $A+B \geq 1$ set $X_{i,T+1}$ to $1$.
        \item If $X_{i,T} = 1$ and $X_{i,T-\delta_I}$ then set $X_{i,T+1}$ to $2$
        \item If $X_{i,T} = 2$ and $X_{i,T-\delta_R}$ then set $X_{i,T+1}$ to $3$
    \end{itemize}

    This process is parameterised by $\theta = (p_E,p_I,\delta_I,\delta_R)$. The first may be interpreted as the
    probabilities of transmission, on any given day, from an exposed node to a susceptible neighbor,
    from an infectious node to a susceptible neighbor. The final two parameters are
    the incubation and recovery periods in days.

    We notice that this process necessarily leads to a state where all nodes are either in state 0 or state 3. This can be used
    as a stopping condition in the iterative process.

    \section{Dynamic Interventions}

    In the SEIR model in the previous section, the Graph, $\mathcal{G}$ was static throughout the state evolution.
    We define a dynamic intervention as those which add additional rules to modify $\mathcal{G}$. Denoting the state of the
    graph at a particular time point by $\mathcal{G}_t$ then an SEIR model with a dynamic intervention modifies the first
    step if the iteration to
    \begin{itemize}
        \item If $X_{i,T} = 0$, simulate $A \sim \text{Binomial}(n_{i,T}(1,\mathcal{G}_T),p_E)$ and
              $B \sim \text{Binomial}(n_{i,T}(2,\mathcal{G}_T),p_I)$, then if $A+B \geq 1$ set $X_{i,T+1}$ to $1$.
    \end{itemize}
    Then adds an additional step at each time point, once the node iteration is complete, which obtains $\mathcal{G}_{T+1}$
    conditional the histories of $\{X_{i,t}\}$ and on $\mathcal{G}_t$ up to time point $T$

    \section{Estimands}

    The desired estimand of interest is the expected peak proportion of infected individuals given an intial graph $\mathcal{G}$
    and dynamic intervention rule $\mathcal{D}$.
    \begin{equation}
        \psi(\mathcal{G},\mathcal{D}) = \e{\max_{t \geq 0 } \{S_t\} }
    \end{equation}
    Such an estimand can be made using the Monte Carlo estimate
    \begin{equation}
        \hat{\psi}(\mathcal{G},\mathcal{D}) = M^{-1} \sum_{j=1}^{M} \big(\max_{t \geq 0 } \{S_t\}\big)_j
    \end{equation}
    where $\big(\max_{t \geq 0 } \{S_t\}\big)_j$ is a set of $M$ simulated samples.

\end{document}